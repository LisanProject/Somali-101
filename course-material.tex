% Options for packages loaded elsewhere
% Options for packages loaded elsewhere
\PassOptionsToPackage{unicode}{hyperref}
\PassOptionsToPackage{hyphens}{url}
\PassOptionsToPackage{dvipsnames,svgnames,x11names}{xcolor}
%
\documentclass[
  a4paper,
  DIV=11,
  numbers=noendperiod]{scrartcl}
\usepackage{xcolor}
\usepackage{amsmath,amssymb}
\setcounter{secnumdepth}{5}
\usepackage{iftex}
\ifPDFTeX
  \usepackage[T1]{fontenc}
  \usepackage[utf8]{inputenc}
  \usepackage{textcomp} % provide euro and other symbols
\else % if luatex or xetex
  \usepackage{unicode-math} % this also loads fontspec
  \defaultfontfeatures{Scale=MatchLowercase}
  \defaultfontfeatures[\rmfamily]{Ligatures=TeX,Scale=1}
\fi
\usepackage{lmodern}
\ifPDFTeX\else
  % xetex/luatex font selection
\fi
% Use upquote if available, for straight quotes in verbatim environments
\IfFileExists{upquote.sty}{\usepackage{upquote}}{}
\IfFileExists{microtype.sty}{% use microtype if available
  \usepackage[]{microtype}
  \UseMicrotypeSet[protrusion]{basicmath} % disable protrusion for tt fonts
}{}
\makeatletter
\@ifundefined{KOMAClassName}{% if non-KOMA class
  \IfFileExists{parskip.sty}{%
    \usepackage{parskip}
  }{% else
    \setlength{\parindent}{0pt}
    \setlength{\parskip}{6pt plus 2pt minus 1pt}}
}{% if KOMA class
  \KOMAoptions{parskip=half}}
\makeatother
% Make \paragraph and \subparagraph free-standing
\makeatletter
\ifx\paragraph\undefined\else
  \let\oldparagraph\paragraph
  \renewcommand{\paragraph}{
    \@ifstar
      \xxxParagraphStar
      \xxxParagraphNoStar
  }
  \newcommand{\xxxParagraphStar}[1]{\oldparagraph*{#1}\mbox{}}
  \newcommand{\xxxParagraphNoStar}[1]{\oldparagraph{#1}\mbox{}}
\fi
\ifx\subparagraph\undefined\else
  \let\oldsubparagraph\subparagraph
  \renewcommand{\subparagraph}{
    \@ifstar
      \xxxSubParagraphStar
      \xxxSubParagraphNoStar
  }
  \newcommand{\xxxSubParagraphStar}[1]{\oldsubparagraph*{#1}\mbox{}}
  \newcommand{\xxxSubParagraphNoStar}[1]{\oldsubparagraph{#1}\mbox{}}
\fi
\makeatother


\usepackage{longtable,booktabs,array}
\usepackage{calc} % for calculating minipage widths
% Correct order of tables after \paragraph or \subparagraph
\usepackage{etoolbox}
\makeatletter
\patchcmd\longtable{\par}{\if@noskipsec\mbox{}\fi\par}{}{}
\makeatother
% Allow footnotes in longtable head/foot
\IfFileExists{footnotehyper.sty}{\usepackage{footnotehyper}}{\usepackage{footnote}}
\makesavenoteenv{longtable}
\usepackage{graphicx}
\makeatletter
\newsavebox\pandoc@box
\newcommand*\pandocbounded[1]{% scales image to fit in text height/width
  \sbox\pandoc@box{#1}%
  \Gscale@div\@tempa{\textheight}{\dimexpr\ht\pandoc@box+\dp\pandoc@box\relax}%
  \Gscale@div\@tempb{\linewidth}{\wd\pandoc@box}%
  \ifdim\@tempb\p@<\@tempa\p@\let\@tempa\@tempb\fi% select the smaller of both
  \ifdim\@tempa\p@<\p@\scalebox{\@tempa}{\usebox\pandoc@box}%
  \else\usebox{\pandoc@box}%
  \fi%
}
% Set default figure placement to htbp
\def\fps@figure{htbp}
\makeatother





\setlength{\emergencystretch}{3em} % prevent overfull lines

\providecommand{\tightlist}{%
  \setlength{\itemsep}{0pt}\setlength{\parskip}{0pt}}



 


\KOMAoption{captions}{tableheading}
\makeatletter
\@ifpackageloaded{tcolorbox}{}{\usepackage[skins,breakable]{tcolorbox}}
\@ifpackageloaded{fontawesome5}{}{\usepackage{fontawesome5}}
\definecolor{quarto-callout-color}{HTML}{909090}
\definecolor{quarto-callout-note-color}{HTML}{0758E5}
\definecolor{quarto-callout-important-color}{HTML}{CC1914}
\definecolor{quarto-callout-warning-color}{HTML}{EB9113}
\definecolor{quarto-callout-tip-color}{HTML}{00A047}
\definecolor{quarto-callout-caution-color}{HTML}{FC5300}
\definecolor{quarto-callout-color-frame}{HTML}{acacac}
\definecolor{quarto-callout-note-color-frame}{HTML}{4582ec}
\definecolor{quarto-callout-important-color-frame}{HTML}{d9534f}
\definecolor{quarto-callout-warning-color-frame}{HTML}{f0ad4e}
\definecolor{quarto-callout-tip-color-frame}{HTML}{02b875}
\definecolor{quarto-callout-caution-color-frame}{HTML}{fd7e14}
\makeatother
\makeatletter
\@ifpackageloaded{caption}{}{\usepackage{caption}}
\AtBeginDocument{%
\ifdefined\contentsname
  \renewcommand*\contentsname{Table of contents}
\else
  \newcommand\contentsname{Table of contents}
\fi
\ifdefined\listfigurename
  \renewcommand*\listfigurename{List of Figures}
\else
  \newcommand\listfigurename{List of Figures}
\fi
\ifdefined\listtablename
  \renewcommand*\listtablename{List of Tables}
\else
  \newcommand\listtablename{List of Tables}
\fi
\ifdefined\figurename
  \renewcommand*\figurename{Figure}
\else
  \newcommand\figurename{Figure}
\fi
\ifdefined\tablename
  \renewcommand*\tablename{Table}
\else
  \newcommand\tablename{Table}
\fi
}
\@ifpackageloaded{float}{}{\usepackage{float}}
\floatstyle{ruled}
\@ifundefined{c@chapter}{\newfloat{codelisting}{h}{lop}}{\newfloat{codelisting}{h}{lop}[chapter]}
\floatname{codelisting}{Listing}
\newcommand*\listoflistings{\listof{codelisting}{List of Listings}}
\makeatother
\makeatletter
\makeatother
\makeatletter
\@ifpackageloaded{caption}{}{\usepackage{caption}}
\@ifpackageloaded{subcaption}{}{\usepackage{subcaption}}
\makeatother
\usepackage{bookmark}
\IfFileExists{xurl.sty}{\usepackage{xurl}}{} % add URL line breaks if available
\urlstyle{same}
\hypersetup{
  pdftitle={Somali 101:Language and Culture Roadmap},
  pdfauthor={Liban Hussein},
  colorlinks=true,
  linkcolor={blue},
  filecolor={Maroon},
  citecolor={Blue},
  urlcolor={Blue},
  pdfcreator={LaTeX via pandoc}}


\title{Somali 101:Language and Culture Roadmap}
\author{Liban Hussein}
\date{2025-06-04}
\begin{document}
\maketitle

\renewcommand*\contentsname{Table of contents}
{
\hypersetup{linkcolor=}
\setcounter{tocdepth}{3}
\tableofcontents
}

\section{Hordhac -- Introduction}\label{sec-intro}

This document serves as a comprehensive guide for learners of the Somali
language, designed to support both self-study and community-based
instruction.

It is a central hub for:

\begin{itemize}
\tightlist
\item
  \textbf{Structured study notes} organized by theme\\
\item
  \textbf{Essential vocabulary} relevant to everyday life\\
\item
  \textbf{Practice exercises} for speaking, listening, reading, and
  writing\\
\item
  \textbf{Cultural and generational engagement}, including Somali
  proverbs and dialogues\\
\item
  \textbf{Progress tracking} for students over the course of their
  learning
\end{itemize}

The course is intended for learners at all levels and is structured as a
series of thematic units that build gradually toward fluency. It is
especially suited for summer intensives or weekend community classes.

\begin{center}\rule{0.5\linewidth}{0.5pt}\end{center}

\begin{tcolorbox}[enhanced jigsaw, left=2mm, colframe=quarto-callout-tip-color-frame, breakable, colback=white, arc=.35mm, colbacktitle=quarto-callout-tip-color!10!white, leftrule=.75mm, title=\textcolor{quarto-callout-tip-color}{\faLightbulb}\hspace{0.5em}{Why Somali Matters}, titlerule=0mm, toptitle=1mm, rightrule=.15mm, coltitle=black, bottomrule=.15mm, opacityback=0, bottomtitle=1mm, toprule=.15mm, opacitybacktitle=0.6]

For many Somali youth raised in the West, the inability to speak their
mother tongue is a source of quiet grief --- a feeling of being
\textbf{cut off from elders, stories, and the untranslatable richness of
home}.

This course is not just about grammar and vocabulary. It's about
\textbf{reconnection}.\\
It's about helping a generation \textbf{recover what was almost lost}
--- and proudly carry it forward.

\end{tcolorbox}

\begin{tcolorbox}[enhanced jigsaw, left=2mm, colframe=quarto-callout-note-color-frame, breakable, colback=white, arc=.35mm, colbacktitle=quarto-callout-note-color!10!white, leftrule=.75mm, title=\textcolor{quarto-callout-note-color}{\faInfo}\hspace{0.5em}{Oral Heritage Lives Through Language}, titlerule=0mm, toptitle=1mm, rightrule=.15mm, coltitle=black, bottomrule=.15mm, opacityback=0, bottomtitle=1mm, toprule=.15mm, opacitybacktitle=0.6]

Somali has been called a \textbf{``nation of poets''} --- not because
every Somali writes books, but because the oral transmission of wisdom,
history, and values has traditionally been \textbf{carried through
language}.\\
Learning Somali is not just learning a language --- it's entering a
cultural archive.

\end{tcolorbox}

\begin{tcolorbox}[enhanced jigsaw, left=2mm, colframe=quarto-callout-caution-color-frame, breakable, colback=white, arc=.35mm, colbacktitle=quarto-callout-caution-color!10!white, leftrule=.75mm, title=\textcolor{quarto-callout-caution-color}{\faFire}\hspace{0.5em}{Language and Identity}, titlerule=0mm, toptitle=1mm, rightrule=.15mm, coltitle=black, bottomrule=.15mm, opacityback=0, bottomtitle=1mm, toprule=.15mm, opacitybacktitle=0.6]

\begin{quote}
When a generation loses its tongue, it also loses its memory.\\
Preserving Somali is preserving identity --- not only for individuals,
but for families, mosques, and communities seeking continuity.
\end{quote}

\end{tcolorbox}

\begin{tcolorbox}[enhanced jigsaw, left=2mm, colframe=quarto-callout-tip-color-frame, breakable, colback=white, arc=.35mm, colbacktitle=quarto-callout-tip-color!10!white, leftrule=.75mm, title=\textcolor{quarto-callout-tip-color}{\faLightbulb}\hspace{0.5em}{Somali Proverb}, titlerule=0mm, toptitle=1mm, rightrule=.15mm, coltitle=black, bottomrule=.15mm, opacityback=0, bottomtitle=1mm, toprule=.15mm, opacitybacktitle=0.6]

\begin{quote}
\textbf{Nin aan dhul marin, dhaayo ma leh.}\\
\emph{He who has not traveled has no eyes.}
\end{quote}

This widely quoted proverb reminds learners that \textbf{exposure
expands understanding} --- just as travel opens the eyes,
\textbf{language opens doors} to culture, connection, and insight.

\end{tcolorbox}

\subsection{Xarafka iyo Laxanka -- Somali Alphabet and
Pronunciation}\label{sec-alphabet-and-pronunciation}

The Somali alphabet uses Latin script with 21 consonants and 5 vowels.
It was standardized in 1972 and is written \textbf{left to right} like
English. Pronunciation is phonetic, meaning words are typically
pronounced as they are spelled.

\begin{center}\rule{0.5\linewidth}{0.5pt}\end{center}

\subsection{Xarafka Af Soomaaliga -- The Somali
Alphabet}\label{xarafka-af-soomaaliga-the-somali-alphabet}

\begin{longtable}[]{@{}
  >{\raggedright\arraybackslash}p{(\linewidth - 4\tabcolsep) * \real{0.1081}}
  >{\raggedright\arraybackslash}p{(\linewidth - 4\tabcolsep) * \real{0.1622}}
  >{\raggedright\arraybackslash}p{(\linewidth - 4\tabcolsep) * \real{0.7297}}@{}}
\toprule\noalign{}
\begin{minipage}[b]{\linewidth}\raggedright
Letter
\end{minipage} & \begin{minipage}[b]{\linewidth}\raggedright
Example
\end{minipage} & \begin{minipage}[b]{\linewidth}\raggedright
Pronunciation Guide
\end{minipage} \\
\midrule\noalign{}
\endhead
\bottomrule\noalign{}
\endlastfoot
ʾ & abuur & Glottal stop --- like a soft pause, similar to \textbf{ء} in
Arabic \\
B & buug & Like \textbf{b} in ``book'' \\
T & tufaax & Like \textbf{t} in ``top'' \\
J & jeer & Like \textbf{j} in ``job'' \\
KH & khamiis & Like Arabic \textbf{خ}, a rough ``kh'' from the throat \\
DH & dhakhtar & A \textbf{voiced retroflex} (tongue curled back), not
found in English \\
R & roob & Rolled or tapped \textbf{r}, like in Spanish \\
S & subax & Like \textbf{s} in ``sun'' \\
SH & sheeko & Like \textbf{sh} in ``shoe'' \\
D & dheer & Like \textbf{d} in ``door'' \\
C & cunto & A voiced pharyngeal sound (like \textbf{ع} in Arabic) \\
G & guri & Like \textbf{g} in ``go'' \\
F & furan & Like \textbf{f} in ``fun'' \\
Q & qalin & Deep \textbf{k} sound from the throat (like \textbf{ق} in
Arabic) \\
K & kabo & Like \textbf{k} in ``kite'' \\
L & lacag & Like \textbf{l} in ``light'' \\
M & meel & Like \textbf{m} in ``man'' \\
N & naag & Like \textbf{n} in ``name'' \\
W & webi & Like \textbf{w} in ``water'' \\
H & hoy & Like \textbf{h} in ``hat'' \\
Y & yar & Like \textbf{y} in ``yes'' \\
\end{longtable}

\begin{center}\rule{0.5\linewidth}{0.5pt}\end{center}

\subsection{Shaqalka -- Somali Vowels}\label{shaqalka-somali-vowels}

Somali has \textbf{5 vowels}: \textbf{A, E, I, O, U}, and each can be:

\begin{itemize}
\tightlist
\item
  \textbf{Short} (e.g., \emph{run} -- ``true/truth'')
\item
  \textbf{Long} (e.g., \emph{ruun} -- ``home/settlement'')
\end{itemize}

Long vowels are marked by doubling the letter (\textbf{aa, ee, ii, oo,
uu}) and are pronounced \textbf{longer and more forcefully}. They can
\textbf{change the meaning} of a word:

\begin{longtable}[]{@{}lll@{}}
\toprule\noalign{}
Short & Long & Meaning (Short vs.~Long) \\
\midrule\noalign{}
\endhead
\bottomrule\noalign{}
\endlastfoot
u & uu & \emph{bur} = flour, \emph{buur} = mountain \\
a & aa & \emph{gal} = enter, \emph{gaal} = non-Muslim \\
i & ii & \emph{bir} = metal, \emph{biir} = increase/multiply \\
\end{longtable}

\begin{center}\rule{0.5\linewidth}{0.5pt}\end{center}

\begin{tcolorbox}[enhanced jigsaw, left=2mm, colframe=quarto-callout-tip-color-frame, breakable, colback=white, arc=.35mm, colbacktitle=quarto-callout-tip-color!10!white, leftrule=.75mm, title=\textcolor{quarto-callout-tip-color}{\faLightbulb}\hspace{0.5em}{Listen and Repeat}, titlerule=0mm, toptitle=1mm, rightrule=.15mm, coltitle=black, bottomrule=.15mm, opacityback=0, bottomtitle=1mm, toprule=.15mm, opacitybacktitle=0.6]

The Somali language is best learned \textbf{by listening}. Repeating
after a native speaker helps train your ear to subtle sounds like
\textbf{C}, \textbf{DH}, and \textbf{KH}.

\end{tcolorbox}

\begin{center}\rule{0.5\linewidth}{0.5pt}\end{center}

\subsection{Laxanka iyo Dhawaaqa -- Notes on
Pronunciation}\label{laxanka-iyo-dhawaaqa-notes-on-pronunciation}

\begin{itemize}
\tightlist
\item
  \textbf{ʾ (Glottal stop)} represents a \textbf{brief pause} or a sound
  from the throat, like a soft catch in the breath --- similar to Arabic
  \textbf{ء} (hamzah).
\item
  \textbf{C} is not like the English ``see'' --- it's a voiced
  \textbf{pharyngeal} sound from the throat (like Arabic \textbf{ع}).
\item
  \textbf{DH} is pronounced by curling the tongue back (retroflex). This
  sound \textbf{does not exist in English}.
\item
  \textbf{KH} is like Arabic \textbf{خ} --- a rough ``kh'' from the
  upper throat.
\item
  \textbf{Q} is pronounced \textbf{deep in the throat} --- not like
  ``k'', but more forceful (like Arabic \textbf{ق}).
\item
  \textbf{R} is lightly rolled, similar to Spanish.
\item
  \textbf{Double vowels (aa, ee, etc.)} are \textbf{longer} and often
  stressed.
\end{itemize}

\begin{center}\rule{0.5\linewidth}{0.5pt}\end{center}

\begin{tcolorbox}[enhanced jigsaw, left=2mm, colframe=quarto-callout-note-color-frame, breakable, colback=white, arc=.35mm, colbacktitle=quarto-callout-note-color!10!white, leftrule=.75mm, title=\textcolor{quarto-callout-note-color}{\faInfo}\hspace{0.5em}{Tip for Arabic Speakers}, titlerule=0mm, toptitle=1mm, rightrule=.15mm, coltitle=black, bottomrule=.15mm, opacityback=0, bottomtitle=1mm, toprule=.15mm, opacitybacktitle=0.6]

If you know Arabic, Somali letters \textbf{C, Q, KH, and DH} will sound
familiar to you. But even native Arabic speakers need to adjust their
ear for \textbf{DH}, which is distinct from Arabic sounds.

\end{tcolorbox}

\subsection{Isbarasho -- Introducing Yourself}\label{unit1-intro}

Isbarasho wanaagsan waxay ku bilaabataa salaan iyo su'aalo aasaasi ah.
Baabkani wuxuu soo bandhigayaa erayada muhiimka ah iyo weedho si aad u
isticmaasho marka aad qof cusub la kulanto.

\emph{A good introduction begins with greetings and basic questions.
This chapter presents essential words and phrases for you to use when
meeting someone new.}

\begin{center}\rule{0.5\linewidth}{0.5pt}\end{center}

\begin{tcolorbox}[enhanced jigsaw, left=2mm, colframe=quarto-callout-tip-color-frame, breakable, colback=white, arc=.35mm, colbacktitle=quarto-callout-tip-color!10!white, leftrule=.75mm, title=\textcolor{quarto-callout-tip-color}{\faLightbulb}\hspace{0.5em}{Start With What You Know}, titlerule=0mm, toptitle=1mm, rightrule=.15mm, coltitle=black, bottomrule=.15mm, opacityback=0, bottomtitle=1mm, toprule=.15mm, opacitybacktitle=0.6]

You don't need to memorize every word immediately.\\
Focus on 5--7 phrases that you can \textbf{say confidently} --- like
your name, where you're from, and how you're doing. Then build from
there.

\end{tcolorbox}

\begin{center}\rule{0.5\linewidth}{0.5pt}\end{center}

\subsection{Kalmado Muhiim ah iyo Weedho -- Key Vocabulary and
Phrases}\label{kalmado-muhiim-ah-iyo-weedho-key-vocabulary-and-phrases}

\begin{longtable}[]{@{}ll@{}}
\toprule\noalign{}
English & Somali \\
\midrule\noalign{}
\endhead
\bottomrule\noalign{}
\endlastfoot
Peace be upon you & Asalaamu Calaykum \\
And peace be upon you & Wa Calaykum Salaam \\
Good morning & Subax wanaagsan \\
How are you? & Sidee tahay / See tahay / Ii waran? \\
I am fine & Waan fiicanahay \\
Welcome & Soo dhowoow \\
What is your name? & Magacaa? \\
My name is \ldots{} & Magacaygu waa \ldots{} \\
Where are you from? & Xagee baad ka timid? \\
I am from Somalia & Waxaan ka imid Soomaaliya \\
Nice to meet you & Kulan wanaagsan \\
Goodbye & Nabadeey / Macsalaamo \\
Teacher & Macallin \\
Student & Arday \\
Phone & Telefoon \\
Address & Adrees / Cinwaan \\
\end{longtable}

\begin{center}\rule{0.5\linewidth}{0.5pt}\end{center}

\begin{tcolorbox}[enhanced jigsaw, left=2mm, colframe=quarto-callout-note-color-frame, breakable, colback=white, arc=.35mm, colbacktitle=quarto-callout-note-color!10!white, leftrule=.75mm, title=\textcolor{quarto-callout-note-color}{\faInfo}\hspace{0.5em}{Cultural Insight}, titlerule=0mm, toptitle=1mm, rightrule=.15mm, coltitle=black, bottomrule=.15mm, opacityback=0, bottomtitle=1mm, toprule=.15mm, opacitybacktitle=0.6]

Somalis often greet with \textbf{a warm tone and repeated blessings}.
Learning to pronounce greetings \textbf{clearly and kindly} is one of
the most respected signs that you're reconnecting with the culture.

\end{tcolorbox}

\begin{center}\rule{0.5\linewidth}{0.5pt}\end{center}

\subsection{Magacyada iyo Erayada Shaqada -- Professions and
Work-Related
Terms}\label{magacyada-iyo-erayada-shaqada-professions-and-work-related-terms}

\begin{longtable}[]{@{}ll@{}}
\toprule\noalign{}
English & Somali \\
\midrule\noalign{}
\endhead
\bottomrule\noalign{}
\endlastfoot
Teacher & Macallin \\
Student & Arday \\
Worker & Shaqaale \\
Carpenter & Farshaxanle \\
Retired & Hawlgab \\
Translator & Turjubaan \\
Accountant & Xisaabiye \\
Unemployed & Shaqo la'aan \\
\end{longtable}

\begin{center}\rule{0.5\linewidth}{0.5pt}\end{center}

\subsection{Isku Day: Tarjum Weedhahan -- Try Translating These
Sentences}\label{isku-day-tarjum-weedhahan-try-translating-these-sentences}

\begin{enumerate}
\def\labelenumi{\arabic{enumi}.}
\tightlist
\item
  Magacaygu waa \ldots{} waxaan ka imid \ldots{}
\item
  Magacaa? (Wiil/Gabar)
\item
  Waxaan ahay Soomaali.
\item
  Waxaan ka imid Mareykanka.
\item
  Xaggee baad ka timid?
\item
  Waa maxay shaqadaada?
\item
  Waxaan ahay macallin.
\item
  Waan fiicanahay, mahadsanid.
\item
  Subax wanaagsan!
\item
  Nabadeey! Waa inoo mar kale haddii uu Ilaahay idmo!
\end{enumerate}

\begin{center}\rule{0.5\linewidth}{0.5pt}\end{center}

\begin{tcolorbox}[enhanced jigsaw, left=2mm, colframe=quarto-callout-important-color-frame, breakable, colback=white, arc=.35mm, colbacktitle=quarto-callout-important-color!10!white, leftrule=.75mm, title=\textcolor{quarto-callout-important-color}{\faExclamation}\hspace{0.5em}{Practice With a Partner}, titlerule=0mm, toptitle=1mm, rightrule=.15mm, coltitle=black, bottomrule=.15mm, opacityback=0, bottomtitle=1mm, toprule=.15mm, opacitybacktitle=0.6]

Language is meant to be spoken. If you can, practice introducing
yourself with a family member, elder, or friend --- even if they answer
in English.\\
Your confidence grows \textbf{with repetition and real interaction.}

\end{tcolorbox}

\begin{center}\rule{0.5\linewidth}{0.5pt}\end{center}

\subsection{Erayo u gaara Qofka -- Personal
Pronouns}\label{erayo-u-gaara-qofka-personal-pronouns}

\begin{longtable}[]{@{}ll@{}}
\toprule\noalign{}
English & Somali \\
\midrule\noalign{}
\endhead
\bottomrule\noalign{}
\endlastfoot
I & Aniga \\
You (m) & Adiga (Wiil) \\
You (f) & Adiga (Gabar) \\
He & Isaga \\
She & Iyada \\
\end{longtable}

\begin{center}\rule{0.5\linewidth}{0.5pt}\end{center}

\subsection{Qabiil iyo Qaranimo -- Nationalities and
Languages}\label{qabiil-iyo-qaranimo-nationalities-and-languages}

\begin{longtable}[]{@{}lll@{}}
\toprule\noalign{}
Country & Nationality & Language \\
\midrule\noalign{}
\endhead
\bottomrule\noalign{}
\endlastfoot
Soomaaliya & Soomaali & Af Soomaali \\
Masar & Masri & Carabi \\
Mareykanka & Mareykan & Ingiriis \\
\end{longtable}

\begin{center}\rule{0.5\linewidth}{0.5pt}\end{center}

\subsection{Sheeko Dheer oo Isbarasho -- Extended Introductory
Dialogue}\label{sheeko-dheer-oo-isbarasho-extended-introductory-dialogue}

\begin{tcolorbox}[enhanced jigsaw, left=2mm, colframe=quarto-callout-caution-color-frame, breakable, colback=white, arc=.35mm, colbacktitle=quarto-callout-caution-color!10!white, leftrule=.75mm, title=\textcolor{quarto-callout-caution-color}{\faFire}\hspace{0.5em}{Click to Practice a Conversation}, titlerule=0mm, toptitle=1mm, rightrule=.15mm, coltitle=black, bottomrule=.15mm, opacityback=0, bottomtitle=1mm, toprule=.15mm, opacitybacktitle=0.6]

\textbf{Qof 1:} Iska warran!\\
\textbf{Qof 2:} Waan fiicanahay. Soo dhowoow!\\
\textbf{Qof 1:} Sidee tahay?\\
\textbf{Qof 2:} Waan fiicanahay, Alxamdulillaah. Adiguna?\\
\textbf{Qof 1:} Magacaa?\\
\textbf{Qof 2:} Magacaygu waa Axmed. Adiga?\\
\textbf{Qof 1:} Waxaan ahay Saara. Xaggee baad ka timid?\\
\textbf{Qof 2:} Waxaan ka imid Muqdisho. Adiguna?\\
\textbf{Qof 1:} Waxaan ka imid Nairobi. Shaqadaada?\\
\textbf{Qof 2:} Waxaan ahay macallin. Adiguna?\\
\textbf{Qof 1:} Waxaan ahay ardayad. Cinwaankaagu?\\
\textbf{Qof 2:} Waa 12 Jidka Muqdisho. Kaaga na?\\
\textbf{Qof 1:} Waa 8 Waddada Wajeer. Nambarkaaga telefoonka?\\
\textbf{Qof 2:} 061-2345678. Kanaga?\\
\textbf{Qof 1:} 061-8765432. Emailkaagu?\\
\textbf{Qof 2:} ahmed@gmail.com. Kaaga na?\\
\textbf{Qof 1:} sarah@email.com.\\
\textbf{Qof 2:} Aad baan ugu faraxsanahay inaan ku kulmo!\\
\textbf{Qof 1:} Aniguna sidoo kale!

\end{tcolorbox}

\begin{center}\rule{0.5\linewidth}{0.5pt}\end{center}

\subsection{Fiiro Gaar ah -- Special
Note}\label{fiiro-gaar-ah-special-note}

Baabkani wuxuu xoogga saarayaa fahamka isbarasho, isticmaalka erayada
aasaasiga ah, iyo kalmadaha shakhsiyeed. Soo koobidda weedho iyo
ficillada maalinlaha ah waxay gacan ka geysanayaan xoojinta xirfadaha
luqadda.

This chapter focuses on understanding introductions, using basic
vocabulary, and personal terms. Summarizing daily expressions and
actions helps reinforce language skills.




\end{document}
